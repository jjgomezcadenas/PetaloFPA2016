\subsubsection*{Personnel resources and requirements of the DET subproject}

Subproject DET will be coordinated by prof. J. D\'iaz and will benefit from the support of the IFIC technical division and the NEXT technical group.

The proposed objectives need a full-time mechanical engineer with expertise in cryogenics to design and build P2, PEP and the gas system. The mechanical engineer will work during the first year in the construction of the P2 detector, and during the second year in the construction of the PEP detector. During the third year, he or she will be in charge of the installation of PEP at the hospital La Fe, and the safety analysis needed for the certification of the apparatus. In addition, he or she will work in the design of a large PETALO scanner, applying the experience and know-how acquired during the first two years of the project.

In addition, a post-doc is needed also for three years. The post-doc will work in the development of SIMPLE and RAP during the first year of the project, will be in charge of the operation and data taking of P2 and PEP (second and third year), will play a leading role in the data analysis of both detectors, and will collaborate with the IMG group in imaging reconstruction.

The DET subproject requires 2 years for the mechanical engineer and 2 years for the post-doc. The project will seek funding for the third year competing to the regular national and international calls for support technicians, Juan de la Cierva, Marie Curie, and/or funds from the Severo Ochoa grant at IFIC.

\subsubsection*{Personnel resources and requirements of the ASIC subproject}

The ASIC subproject is based in the expertise existing  at the I3M/UPV, which includes the to co-PIs, Prof. V. Herrero Bosch (VHB), Prof. Rafael Gadea (RG).
In addition the group has  one student  currently working with VHB.

The project requires an electronic engineer for two years. He or she will be in charge of the construction and testing of the DBs for P2 and PEP, the setting up and commissioning of the DAQ and the development of the slow controls. In addition, the engineer will work in the characterisation of the electronics in cryogenics conditions. He or she will be closely supervised for all the above tasks by VHB and RG.

\subsubsection*{Personnel resources and requirements of the IMG subproject}
 The IMG subproject includes an interdisciplinary team, lead by Dr. Irene Torres, a nuclear physicist with a Ph.D. in medical imaging.

 The project needs a post-doc for two years. He or she will play a leading role in the development of imaging algorithms, reconstruction and integration of image post-processing and the extraction of imaging biomarkers.

The post-doc background will be focused in the field of biomedical engineering and medical physics, specially in medical imaging acquisition technologies and in medical imaging processing. He or she will start developing imaging reconstruction algorithms (see the methodology section) during the second year of the project using Monte Carlo Data and will continue during the third year using PEP data. The post-doc will also work in the assessment of the imaging reconstruction capabilities of a large PETALO-TOF scanner.

%The tasks to be addressed by the Post-Doc will be under the frame of the integration of the PET solution with the NMR, the requirements of the new technology and the aspects of image reconstruction and processing. These will include:
% \begin{enumerate}
%\item Definition of MRI compatibility requirements and clinical spatial resolution.
%\item Study and modelling of image degradation phenomena LXSC-PET.
%\item Mathematical modelling of signals from PET detectors.
%\item Integration in the software of the user interface of the system.
%\item Development of hybrid image reconstruction methods for PET-MRI.
%\item Development of algorithms for the extraction of imaging biomarkers.
%\item Integration of imaging biomarkers analysis in user interface.
%\end{enumerate}

\subsubsection*{Personnel costs}
The total personnel costs are computed assuming a total cost of 40,000 \euro\ to the project per post-doc or engineer. The total number of post-doc (engineer) years is 8, resulting in a proposed personnel costs of 320,000 \euro.
