Xenon is a noble gas which scintillates in response to the ionising radiation. The scintillation is very fast and very intense, thus suitable for medical imaging.

This project will develop in two overlapping stages. First, we will conduct an exhaustive investigation of the physics of scintillation in liquid xenon (LXe), and we will characterise the performance of a new type of detection cell, the Liquid Xenon Scintillating Cell (LXSC), a box filled with LXe and readout by silicon photomultipliers (SiPMs). Monte Carlo studies indicate that the LXSC can achieve excellent energy and spatial resolution, as well as superb coincidence resolving time (CRT).  These capabilities will be studied experimentally during the first stage of the project, using an experimental setup called P2. The setup will also be used to study the performance of very fast detectors and electronics suitable for the detection of Cherenkov light in LXe, an activity that will be carried out in close collaboration with project CLUES, also submitted to this call. 

The second stage of this project will be the construction, commissioning and operation of a demonstrator of a new type of PET apparatus, called
PETALO (Positron Electron TOF Apparatus using Liquid xenOn) based on the LXSC. The excellent expected performance of the cell 
makes it possible to build a PET of excellent energy resolution (12-14\% FWHM), very good spatial resolution
(of the order of 2 mm FWHM {\em in the three coordinates}) and excellent time resolution, with a CRT of about 100 ps. Furthermore, LXe is a continuous medium, allowing a very compact detector, with minimal dead areas. Cryogenic operation of SiPMs, permits suppressing to negligible levels one of the most important noise sources of this type of sensors (dark current). Last, but not least, xenon is much cheaper than LYSO (3 \euro/cc to be compared with 40--50 \euro/cc). All this factors put together result in the possibility of building a large PET scanner of competitive cost and high sensitivity (``full body'' PET, with an axial length at least 5 times larger than current devices), with excellent TOF capabilities. 

This projects coordinates three groups whose combined experience permits the construction and commissioning of the referred demonstrator, called PEP (PEtalo Prototype). The DET subproject (IFIC) is in charge of the construction of P2 and PEP. The ASIC subproject (I3M/UPV) is in charge of sensors, front-end electronics and DAQ. The IMG subproject
(GIBI239/LaFe) is in charge of image reconstruction, as well as operation and certification of PEP in a clinical environment. Project CLUES is lead by  the group ICCUB (Instituto de Ciencias del Cosmos,UB).


