This project proposes a proof-of-concept for a new type of PET apparatus, called
PETALO (Positron Electron TOF Apparatus using Liquid xenOn). The active target is liquid xenon (LXe), read by silicon photomultipliers (SiPMs). The basic element of PETALO is the Liquid Xenon Scintillating Cell (LXSC). The simplest design of such a cell is a box in which one of its faces (the exit face w.r.t. the fly direction of the gammas) is instrumented, while the other faces are covered with reflecting Teflon sheets. The best ratio in cost to performance is obtained with the LXSC2, which instruments  the entry and exit faces of the box, achieving excellent energy resolution and spatial resolution (in the three coordinates), as well as superb coincidence resolving time (CRT).  

Xenon is a noble gas which scintillates as response to the ionising radiation. The scintillation is very fast (it is mediated by two constants, una of 2.2 ns and the other one of 27 ns) and very intense (30,000 photons per 511 keV gamma). The combination of both features results in the possibility of building a PET of excellent energy resolution (better than 12-14\% FWHM, which is the same order as the top-of-the-line present LYSO scanners), very good spatial resolution
(of the order of 2 mm FWHM {\em in the three coordinates})  and excellent time resolution, with a CRT of about 100 ps. Furthermore, LXe is a continuous medium, allowing a very compact detector, with minimal dead areas. Cryogenic operation of SiPMs, on the other hand, permits suppressing to negligible levels one of the most important noise sources of this type of sensors (dark current). Last, but not least, xenon is much cheaper than LYSO (3 \euro/cc to be compared with 40--50 \euro/cc). All this factors put together result in the possibility of using the technology to build a large PET scanner of competitive cost and high sensitivity (``full body PET, with an axial length at least 5 times larger than current devices), based in LXe, with excellent TOF capabilities. The use of time of flight to reduce imaging errors is known since the beginning of the technology and the last few years have witnessed a rekindled interest in the technique, including the introduction in the market of a commercial model boasting a CRT of 600 ps (the GEMINI, manufactured by Philips). Other recent devices feature improved CRT near 400 ps. PETALO could achieve a CRT in the range of 100 ps, thus representing a break-through in the technology.  

This projects coordinates three groups whose combined experience permits the construction and commissioning of a demonstrator of PETALO called PEP (PEtalo Prototype). The DET subproject (IFIC) is in charge of the construction  of the detector. It is also in charge of the development of the simulation and software framework, and will lead the study of PETALO as a TOF device. The ASIC subproject (I3M/UPV) is in charge of sensors, front-end electronics and DAQ. The subproject IMG 
(GIBI239/LaFe) is in charge of image analysis, as well as operation and certification of PEP in a clinical environment.   



