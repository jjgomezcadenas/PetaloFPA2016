\subsubsection*{Overall objectives}
The major objectives of this research proposal are: 
\begin{enumerate}
\item {\bf An exhaustive investigation of the physics of scintillation in liquid xenon}. This includes a precise measurement of the yield and the time structure (decay constants) of scintillation for 511 keV gammas. The rationale for such measurements is to improve the current state of the art, in particular given the fact that some of the most relevant measurements are old and affected by relatively large systematics \footcite{Kubota79}. 
\item {\bf Resolution of the LXSC}: this includes energy, spatial and time resolution for LXSCs instrumented with VUV-sensitive SiPMs and for LXSCs instrumented with conventional SiPMs coated with TPB.
\item {\bf Cryogenic operation of electronics}: in order to improve performance and decrease complexity, it would be desirable to use front-end electronics capable to operate at LXe temperatures ($\sim$ 160 K). We will carry an R\&D program to study the performance of the chosen (commercial) electronics at 160 K and will collaborate with the manufacturer (PETSYS) to study the feasibility of a cryogenic version of their current ASICs. 
\item {\bf Studies of fast detectors and fast electronics for the detection of Cherenkov light in LXe}: This last objective will be carried out in collaboration with a group of researchers from the University of Barcelona \footnote{Group ICCUB (Instituto de Ciencias del Cosmos de la UB, icc.ub.edu), project CLUES, (Cherenkov Light Ultrafast Electronics and Sensor) submitted to FPA call, 2016.}
\item {\bf Demonstration of the PETALO concept}: in particular by demonstrating its TOF capabilities.   
\end{enumerate}

This objectives will be achieved through two overlapping stages:

\begin{enumerate}
\item {\bf P2}. During the first  12 months of the project, an experimental setup, called P2, will be built. P2 will be used during the next 18 months to perform the measurements corresponding to the first four objectives.  
\item {\bf PEP}. A full demonstrator of PETALO, called PEP, deploying 14 LXSCs will be used to prove the feasibility of a full-body, high-sensitivity, TOF-capable LXe scanner. The cryostat of PEP and the 14 LXSCs that instrument it will be built during the second year of the project, and the system will be commissioned during the first 4 months of the third year and operated during the last 8 months of the project.   
\end{enumerate}

The objectives presented in this project are very well aligned to the Spanish program for science. They involve a direct technology transfer from basic science in particle physics to a medical imaging application of high social impact (Reto 1: Salud). It involves a multidisciplinary collaboration between physicist, engineers and medical doctors. And it has a large potential for generating commercial spinoffs, including the possibility of a break through in  PET-TOF, and PET-FBP technology.  
