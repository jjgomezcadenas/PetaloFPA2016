%C.3. IMPACTO ESPERADO DE LOS RESULTADOS 

The expected impact of this project is very large. A successful proof of concept of PETALO would open the way to the subsequent construction of clinical PETs. PETALO offers the possibility to build a full-body, high sensitivity, TOF capable scanner at a very competitive cost. 
%many advantages over conventional SSDs based devices, including a much better energy resolution, true 3D reconstruction that minimises parallax effects and the capability to handle Compton interactions, which are much better identified in the LXSC than in conventional SSDs. In addition, PETALO will be designed, from the beginning, as a fully compatible device with NMR, including both hardware components and the development of the imaging software. On top of the above, PETALO offers the potential of a breakthrough in TOF-PET technology. Last, but not least, the cost of the detection material (LXe) is much lower than the cost SSDs such as LSO and the cost of the SiPMs is decreasing exponentially. Therefore, by developing suitable front-end electronics and DAQ, PETALO could become a very economical PET solution, appropriated for a future large-scale (full-body) PET.

This project will also advance the field in several other aspects, including: a) a more detailed understanding of the physics of liquid xenon; b) the potential to develop electronics suitable to work at 160 K; a collaborative effort with project CLUES (group ICCUB) to advance the field in the area of ultra-fast detectors and electronics. In particular, this project will allow the application of such detectors and electronics to the study of the fast signals produced by scintillation and Cherenkov light in liquid xenon. 

Each subproject in this coordinated project contributes in a decisive way to the impact of PETALO. Subproject DET focuses in transferring the know-how developed at IFIC by basic science experiments such as NEXT to an application of enormous importance for public health. Subproject ASIC provides the expertise in sensors, front-end electronics, data acquisition and ASIC development. Subproject IMG will bring the necessary expertise in image reconstruction and will be in charge of transferring the PEP detector to the clinical environment. 

PETALO has already produced a patent request, and a number of other patents will certainly follow. The apparatus can be clearly commercialised, given its many advantages over conventional PETs. The diffusion of the results will proceed through publications in journals and participation in national and international conferences. Once the proof of concept is operational, the research team will work actively with companies to explore join ventures and possibilities of transferring to the industrial sector. 
