\subsubsection*{Overall objectives}
The overall objectives of this research proposal are: a) an exhaustive investigation of the physics of scintillation in liquid xenon and its potential for medical imaging. Specifically, we propose:
\begin{enumerate}
\item A precise measurement of the yield of 511 keV photons and its time dependence, two crucial parameters for PET performance which are still today affected of considerable systematic uncertainties.
\item A systematic measurement of the performance of the LXSC (including energy resolution, spatial resolution and CRT). Two configurations will be studied, one with two planes of SiPMs per cell (LXSC2) and another with a single plane (LXSC1), in order to assess quantitatively the tradeoff between cost and performance.
\item A systematic measurement of the performance of different SiPM in LXe. We will study the dependence of the time resolution with the SiPM capacitance, the reduction of dark current, and measure the CRT that can be achieved with VUV--sensitive SiPMs and with SiPMs coated with TPB. 
\item A systematic evaluation of state-of-the-art custom ASIC electronics for TOF-PET. Specifically we will study the performance of the TOF PET ASIC from PETSYS \footnote{http://www.petsyselectronics.com/web/public/products/1}. 
\item We also intent to collaborate with a group of researchers from UB \footnote{Gascon, Graciani, Garrido} to evaluate other alternatives, such as FlexToT \footcite{Trenado:2014vba}. 
\item We will study the eventual operation of PETSYS and/or FlexToT electronics in LXe cryogenic conditions.
\item In collaboration with the UB group, we will study the performance of fast detectors (such as MCPs or SPADs) with fast custom electronics (provided by the UB group) with the aim of detecting and exploiting for CRT measurements the Cherenkov light produced in xenon.  
\item Last but not least, we propose the construction, commissioning and operation of a prototype of PETALO, which we call PEP (Petalo Prototype). PEP will be built in three stages. Stage 1 (P1) will operate a LXSC in a small cryostat. The gas system and the common cryogenic elements will be developed, and studies of yield, timing, energy and spatial resolution will be carried out, as well as studied to characterise SiPMs and electronics.  Stage 2 (P2), will add a second LXSC for detailed CTR measurements. Finally, we will build the full PEP demonstrator equipped with 15 LXSC. 
\end{enumerate}

The objectives presented in this project are very well aligned to the Spanish program for science. They involve a direct technology transfer from basic science in particle physics to a medical imaging application of high social impact (Reto 1: Salud). It involves a multidisciplinary collaboration between physicist, engineers and medical doctors. And it has a large potential for generating commercial spinoffs, including the possibility of a break through in  PET-TOF, and PET-FBP technology.  
