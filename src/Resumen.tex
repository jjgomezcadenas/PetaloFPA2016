El xenón es un gas noble, que centellea en respuesta a la radiación ionizante. La señal de centelleo del xenón líquido es muy rápida y muy intensa, siendo por tanto potencialmente útil para la imagen médica.

Este proyecto se desarrollará en dos fases concurrentes. En primer lugar, se realizará una investigación 
exhaustiva de la física del xenón líquido (LXe de sus siglas en inglés) y se caracterizarán las prestaciones de un nuevo tipo de celda de detección, llamada la Celda Centelleante de Xenón Líquido (LXSC), cuyo diseño básico es el de una caja llena con LXe y leída por fotomultiplicadores de silicio (SiPMs). Recientes estudios de
Monte Carlo indican que la LXSC puede conseguir una excelente resolución espacial y de energía, al igual que un tiempo de coincidencia (``coincidence resolving time'', o CRT) muy bueno. Estas capacidades serán estudiadas
experimentalmente durante la primera fase del proyecto, usando un dispositivo experimental denominado P2. El mismo dispositivo será también utilizado para estudiar las prestaciones de nuevos detectores y electrónica muy rápidos, con la posible aplicación de la detección de la luz de Cherenkov and LXe. Esta actividad se realizará en estrecha colaboración con el proyecto CLUES, que también se ha presentado a esta convocatoria. 

La segunda fase de este proyecto propone la construcción, puesta a punto y operación de un prototipo de un nuevo tipo de aparato PET denominado PETALO (Positron Electron TOF Apparatus using Liquid xenOn) basado en la LXSC. Las excelentes prestaciones de la LXSC hacen posible un PET de gran resolución en energía (12--14\% FWHM), buena resolución espacial (del orden de 2 mm FWHM {\em en las tres coordenadas}) y excelente resolución temporal, con un CRT de unos 100 ps. Además, el LXe es un medio continuo, lo que permite diseñar un detector muy compacto en el que se minimizan las áreas ciegas. La operación criogénica de los SiPMs permite eliminar completamente una de las fuentes de ruido más importantes de estos detectors, la corriente oscura. Por último, el coste del xenón líquido es muy inferior al LYSO (3 \euro/cc, a comparar con 40-50 \euro/cc). Todos estos factores se combinan para ofrecer la posibilidad de construir un detector de gran tamaño y alta sensibilidad (``PET de cuerpo completo'', con una longitud axial del orden de cinco veces la de los actuales dispositivos) basado en el LXe, con excelentes prestaciones para tiempo de vuelo (TOF).  

Este proyecto coordina tres grupos cuya experiencia combinada permiten la construcción y puesta a punto del prototipo mencionado,  llamado PEP (PEtalo Prototype). El subproyecto DET (IFIC) está a cargo de la construcción del aparato. así como de la construcción de P2.  El subproyecto IMG (GIBI239/LaFe) está a cargo del análisis de imagen y la operación y certificación de PEP en entorno clínico. EL proyecto CLUES está liderado por el grupo ICCUB (Instituto de Ciencias del Cosmos,UB).



