Este proyecto propone una prueba de concepto para un nuevo tipo de aparato PET, denominado PETALO (Positron Electron TOF Apparatus using Liquid xenOn). El material activo es xenón líquido leído por fotomultiplicadores de silicio (SiPM). El elemento básico de PETALO es la celda centelleadora de xenón líquido (LXSC).  El diseño más simple de dicha celda es una caja en la que una de sus caras (la cara de salida con respecto a la dirección de vuelo de las gammas) está instrumentada, mientras que el resto se recubre con láminas de Teflón. La configuración que ofrece la mejor relación calidad--coste es la LXSC2, donde se instrumentan las caras de entrada y salida de la caja con SiPMs, consiguiendo con ello una excelente resolución en energía, resolución espacial (en las tres coordenadas) y resolución en coincidencia temporal (CRT de sus siglas en inglés). 

El xenón es un gas noble, que centellea en respuesta a la radiación. La señal de centelleo del xenón líquido es muy rápida (caracterizada por dos constantes una de 2.2 ns y otra de 27 ns) y muy intensa (30,000 fotones por gama de 511 keV). La combinación de ambas características hace posible utilizar el xenón líquido para construir un PET de gran resolución en energía (alrededor del 12--14\% FWHM, una resolución comparable con los mejores PET actuales, basados en LYSO), buena resolución espacial (del orden de 2 mm FWHM {\em en las tres coordenadas}) y excelente resolución temporal, con un CTR de unos 100 ps. Además, el LXe es un medio continuo, lo que permite diseñar un detector muy compacto en el que se minimizan las áreas ciegas. La operación criogénica de los SiPMs permite eliminar completamente una de las fuentes de ruido más importantes de estos detectors, la corriente oscura. Por último, el coste del xenón líquido es muy inferior al LYSO (3 \euro/cc, a comparar con 40-50 \euro/cc). Todos estos factores se combinan para ofrecer la posibilidad de construir un detector de gran tamaño y alta sensibilidad (``PET de cuerpo completo'', con una longitud axial del orden de cinco veces la de los actuales dispositivos) basado en el LXe, con excelentes prestaciones para tiempo de vuelo (TOF).  La aplicación de tiempo de vuelo para reducir los errores de reconstrucción de imagen en PET, conocida desde el albor de la tecnología ha resurgido con fuerza en los últimos años, incluyendo la introducción en el mercado de un modelo comercial con un CRT de 600 ps (el GEMINI de Philips). PETALO aspira a una resolución temporal en el rango de los 100 ps, lo que supondría un salto cuantitativo en la tecnología.

Este proyecto coordina tres grupos cuya experiencia combinada permiten la construcción y puesta a punto de un demostrador de PETALO, llamado PEP (PEtalo Prototype). El subproyecto DET (IFIC) está a cargo de la construcción del aparato. También desarrollará la simulación y el entorno de software para la operación del detector y se ocupará del estudio de su capacidad como PET-TOF. El subproyecto ASIC (I3M/UPV) está a cargo de los sensores, la electrónica y la adquisición de datos.  El subproyecto IMG (GIBI239/LaFe) está a cargo del análisis de imagen y la operación y certificación de PEP en entorno clínico. 



